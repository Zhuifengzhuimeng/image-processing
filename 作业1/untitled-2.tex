\documentclass[withoutpreface,bwprint]{cumcmthesis}
%这里是导言区
\usepackage{url}
\usepackage{epsfig}
\usepackage{graphicx}
\usepackage{amsmath}
\usepackage{diagbox}
\usepackage{algorithm}
\usepackage{listings}
\usepackage{algpseudocode}
\usepackage{appendix}
\usepackage{mathrsfs}
\title{图像处理中的数学方法 hw1}
\date{\today}
\author{Jinze Wu 1700010643}
\begin{document}
\maketitle

\section{作业内容1}
梯度的离散逼近:
\begin{equation}
\left.\frac{\partial u}{\partial x}\right|_{i,j}\approx \lambda \frac{v_{i+1,j}-v_{i-1,j}}{2h}+
\frac{1-\lambda}{2}(\frac{v_{i+1,j+1}-v_{i-1,j+1}}{2h}+\frac{v_{i+1,j-1}-v_{i-1,j-1}}{2h})
\end{equation}

左图中计算得
$ \nabla v=(1/2,0)$,$\lVert \nabla u \rVert_{2}=\frac{1}{2}$

右图中计算得
$\nabla v=((1+\lambda)/4,(1+\lambda)/4)$,$\lVert \nabla u \rVert_{2}=\frac{1+\lambda}{4}\sqrt{2}$

左右两边相等时$\lambda=\sqrt{2}-1$

Laplace的离散逼近
\begin{equation}
\begin{split}
\left.\triangle v\right|_{i,j}\approx \lambda \frac{v_{i+1,j}+v_{i-1,j}+v_{i,j+1}+v_{i,j-1}-4v_{i,j}}{h^{2}}+\\
(1-\lambda)(\frac{v_{i+1,j+1}+v_{i-1,j+1}+v_{i+1,j-1}+v_{i-1,j-1}-4v_{i,j}}{2h^{2}})
\end{split}
\end{equation}

左图中计算得
$ \triangle v=1$

右图中计算得
$\triangle v=\frac{1+3\lambda}{2}$

左右两边相等时$\lambda=\frac{1}{3}$

\section{作业内容2}
\subsection{问题描述}
用以下模型对图片进行去噪去模糊
\begin{equation}
\hat{u}=argmin_{u}\lambda \int \lvert \nabla u\rvert dx+\frac{1}{2}\int (\mathcal{A}u-f)^{2}dx
\end{equation}

其中 f 是待去噪去模糊的图片, u 是优化变量, 得到的最优解 û 为处理结果. A 是以 k
为卷积核的卷积算子:

求解方法:对(3)式采用Alternative Direction Minimization of Multipliers (ADMM)算法

\begin{equation}
\mathcal{A}u=k\ast u
\end{equation}
\subsection{求解方法}

注:以下离散均选取循环边值条件,步长h=1

先将原问题离散:

TV的离散:
\begin{equation}
\int \lvert \nabla u\rvert dx=\sum_{i,j=1}^N\lvert \left.\nabla u\right|_{i,j}\rvert
\end{equation}

其中梯度的离散

\begin{equation}
\left.\nabla u\right|_{i,j}\approx (u_{i+1,j}-u_{i,j},u_{i,j+1}-u_{i,j})
\end{equation}

\begin{equation}
\int (\mathcal{A}u-f)^{2}dx=\sum_{i,j=1}^N\lvert \mathcal{A}u\rvert_{i,j}-f\rvert_{i,j} \rvert^{2}
\end{equation}

再将$u_{i,j}\quad i,j=1,2,\dots,N$向量化为一个$N^2\times 1$的向量,对应的也将f向量化

设差分梯度算子$\nabla u=(W_{1}u,W_{2}u)$其中$W_{1},W_{2}$是$N^2\times N^2$矩阵,卷积算子$\mathcal{A}$对应于将$N^2\times N^2$矩阵A左乘u,即得:
\begin{equation}
\int \lvert \nabla u\rvert dx =\lvert Wu\rvert_{tv} \quad \int (\mathcal{A}u-f)^{2}dx=\lvert Au-f\rvert_{2}
\end{equation}

则原问题转化为:

\begin{equation}
\hat{u}=argmin_{u}\lambda \lvert Wu\rvert_{tv}+\frac{1}{2}(\mathcal{A}u-f)^{2}
\end{equation}
再对离散问题采用Alternative Direction Minimization of Multipliers (ADMM)算法:

选取初值$d_{0},b_{0}$,参数$\mu ,\delta$,每步迭代
\begin{equation}
\begin{split}
&u_{k+1}=(A^TA+\mu W^TW)^{-1}(A^tf+\mu W^T(d_{k})-b_{k})\\
&d_{k+1}=P_{\lambda /\mu}(Wu_{k+1}+b_{k}) \quad P_{\lambda}{v}=\frac{v}{\lvert v\rvert}max\{ \lvert v\rvert-\lambda,0\}\\
&b_{k+1}=b_{k}+\delta(Wu_{k+1}-d_{k+1})
\end{split}
\end{equation}

而令
\begin{equation}
D=
\begin{bmatrix}
-1     & 0 & \cdots & 0 &1\\
1      &-1 & \cdots & 0 &0\\
        & 1 & \ddots & \vdots& \vdots\\
\vdots      & \vdots &    &-1  &0\\
0      &0  &\cdots  &1 &-1
\end{bmatrix}
\end{equation}

则$W_{1}=D\otimes I$,$W_{2}=I\otimes D$
\subsubsection{迭代方法}
\subsubsection{共轭梯度法}
\subsubsection{快速Fourier变换}
\subsection{边值条件}
\subsubsection{循环卷积条件}

\subsection{实验结果}

\subsection{结果分析}
\end{document}